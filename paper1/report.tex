\documentclass[sigconf]{acmart}

\usepackage{hyperref}

\usepackage{endfloat}
\renewcommand{\efloatseparator}{\mbox{}} % no new page between figures

\usepackage{booktabs} % For formal tables

\settopmatter{printacmref=false} % Removes citation information below abstract
\renewcommand\footnotetextcopyrightpermission[1]{} % removes footnote with conference information in first column
\pagestyle{plain} % removes running headers

\begin{document}
\title{Big Data Applications in Improving Patient Care}


\author{Janaki Mudvari Khatiwada}
\orcid{}
\affiliation{%
  \institution{University of Indiana}
  \streetaddress{}
  \city{Bloomington} 
  \state{Indiana} 
  \postcode{47408}
}
\email{jmudvari@iu.edu}




\begin{abstract}
 This paper will broadly identify the applications of big data for improving patient care. It will explore how service providers in health-care industries are using big volume of health related data that are generated when patients provide information about their family history, medical history, food and exercise habit or results from clinical tests. It will be an overview of ongoing practices on how big volume of health care data be an important resource for better in-patient and out-patient care.  
\end{abstract}

\keywords{Big Data, Health Care, Patient care Electronic health records}

\maketitle
\section{Introduction}
Health care is one of the service sector where service providers claim to have provided consumers with the best experience possible, whereas consumers are always browsing for the best care facilities that they could possibly get which might save them time and money and have a quality of life. Health service providers collect high volume of information from the consumers every time they visit the facilities. The volume of health related informations generated in a high velocity of time is what consist of big data in health care sector.
These informations besides clinical records can be anything related to a person. Such as  person's ethnic background, exercise routine
and the time he/she spends on it on a weekly or daily basis, general daily meal the person intakes, records on wearable health devices 
and so on. In today's world big data has become very impactful in policy making, solving problems and making prediction on whole range 
of areas and health care has become one of the most important sector to make of use of big data. Big data provides helpful insights 
for prevention, prediction, diagnosis and identification of best treatment option among all, on the basis of insurance plan a person 
has. Clinical practitioners require, share, compare and analyze big data trend to make their medical diagnosis, treatment
recommendation, and prognosis. A richer set of near-real-time information can greatly help
physicians determine the best course of action for their patients, discover new treatment
options, and potentially save lives~\cite{www-hpe}. So to speak fields big data applications in health care for the purpose of 
improving patient care is wide; disease prevention and management, health education, research and development, prognosis  
information sharing, public and individual health management, medical optimization.
Consumers on the other hand use service provider's websites or web-pages to have an insight of the facilities and the physicians.
We look for the ratings and reviews in general based upon which we choose the facility and physician based on other people's experiences for best possible outcome.


\section{Applications}

Health data are stored as electronic medical records(EMR),electronic health records(EMR) or any unstructured records, which are analyzed and shared among clinicians. These data are near real time data. The EHR, being adopted in many countries, offers a source of data the depth of which is almost inconceivable. About 500 petabytes of data was generated by the EHR in 2012, and by 2020, the data will reach 25,000 petabytes\cite{www-ghdonline-org}. One of the trending example of application of big data in tackling opioid crisis in US.
Data scientists at Blue Cross Blue Shield have started working with big data experts at Fuzzy Logix to tackle the problem. Using years of insurance and pharmacy data, Fuzzy Logix analysts have been able to identify 742 risk factors that predict with a high degree of accuracy whether someone is at risk for abusing opioids\cite{www-datapine-com}.

In general, applications of big data in health care for improving patient care can be categorized into following categories: Prevention, Prediction, Diagnosis, Disease Management and Research and Development.

\subsection{Prediction}

Analysis of available health records help make prediction which ultimately benefits general population. Making predictions is one of the most useful applications of big data.Researchers use analysis of medical records to make prediction of patients at risk to a disease. The United States National Institutes of Health has a project known as Pillbox, in which big data are used through the National Library of Medicine\cite{www-tandf-com}. Johns Hopkins University (Baltimore, MD, USA) developed a disease prediction system using the social media service Twitter\cite{www-ncbi-nlm-nih-gov}
The Seton Healthcare Family (Austin, TX, USA) and IBM Joint Development Program have analyzed and tracked medical information, and have predicted outcomes of two million patients per
year\cite{www-uhcjsc-com}. Prediction models are especially useful in explaining epidemics and finding the best approach to deal with it. This helps in population health management. Optum Labs has collected EHRs of over 30 million patients to create a database for predictive analytics tools that will help doctors make big 
data-informed decisions to improve patients treatment\cite{www-mapr-com}.


\subsection{Prevention}

The mantra, "Prevention is always better than cure" is what everybody
wants to implement. Till now physicians have been studying the general pattern of people's lifestyle and make a recommendation on keeping as it is or make a change to prevent their patients from any health problems.Big data help them identify vulnerable population and raise awareness. For example, physicians recommend general public to watch their weight in order to prevent them from diabetes and heart disease. Another such example is, physicians have identified certain population of certain race are more prone to skin cancer when exposed to sun's ultraviolet rays while other race is more prone to have breast cancer. So, they raise awareness and make needed recommendations accordingly. This in totality help make general public's life better and help them live longer and healthy life. Now we have smart-phones and wearables to track our fitness in general, which generate huge volume of data at a high velocity. In the near future, physicians might be using these data to have an understanding of the trend and prepare them for necessary remedies. Often by partnerships between medical and data professionals, with the potential to peer into the future and identify problems before they happen\cite{www-forbes-com}. One recently formed example of such a partnership is the Pittsburgh Health Data Alliance - which aims to take data from various sources (such as medical and insurance records, wearable sensors, genetic data and even social media use) to draw a comprehensive picture of the patient as an individual, in order to offer a tailored healthcare package\cite{www-forbes-com}.   

 

\subsection{Diagnosis}
Early diagnosis of a disease helps in early intervention of disease management thereby saving lives and reducing costs. Prediction models developed by researchers by using big data help in early diagnosis.Predictive modeling over data derived from EHRs is being used for early diagnosis and is reducing mortality rates from problems such as congestive heart failure and sepsis\cite{www-mapr-com}


\subsection{Disease Management} 
Wearable sensors, monitors and other smart devices help both caregivers and patients to keep track of any changes in factors that is affecting their health. Processing real-time events with machine learning algorithms can provide physicians with insights to help them make lifesaving decisions and allow for effective interventions\cite{www-mapr-com}.Ideally, individual and population data would inform each physician and her patient during the decision-making process and help determine the most appropriate treatment option for that particular patient\cite{www-link-springer-com}



\subsection{Research and Development}
Through research of big data from past help physicians identity general variables responsible for illnesses. After identifying general trend, they can make precise recommendation to their patients and thereby help them have a quality of life and save them costs. 
Research and development is one of the important applications of big data and
analytics that helps in finding new tools, more effective medications, drugs 
and treatment regimen.Data-sharing arrangements between the pharmaceutical giants has led to breakthroughs such as the discovery that desipramine, commonly used as an anti-depressant, has potential uses in curing types of lung cancer\cite{www-forbes-com}. Big data helps Pharmaceuticals reduce cost of research and therefore lowers drugs cost which benefits patients. 
 
 




\section{Challenges}
While big healthcare data and applications and analytics provides a huge opportunity in improving patient care, it equally comes with some challenges. 
Privacy and security of personal information is one of the biggest challenge.
In February, the largest ever healthcare-related data theft took place, when hackers stole records relating to 80 million patients 
from Anthem, the second largest US health insurer. Fortunately they only took identity information such as names and addresses, 
and details on illnesses and treatments were not exposed\cite{www-forbes-com}.
Since healthcare data are large in volume and are in variety of forms; structured or unstructured, managing this big data of such variety is a challenge. 




\section{Conclusions}







\appendix



\begin{acks}

I would like to acknowledge prof.Gregory Von Laszweski, TAs who helped me throughout my writing.  
\end{acks}

\bibliographystyle{ACM-Reference-Format}
\bibliography{report} 

\end{document}
