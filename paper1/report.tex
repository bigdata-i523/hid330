\documentclass[sigconf]{acmart}

\usepackage{hyperref}

\usepackage{endfloat}
\renewcommand{\efloatseparator}{\mbox{}} % no new page between figures

\usepackage{booktabs} % For formal tables

\settopmatter{printacmref=false} % Removes citation information below abstract
\renewcommand\footnotetextcopyrightpermission[1]{} % removes footnote with conference information in first column
\pagestyle{plain} % removes running headers

\begin{Paper1}
\title{Big Data Applications in Improving Patient Care}


\author{Janaki Mudvari Khatiwada}
\orcid{}
\affiliation{%
  \institution{University of Indiana}
  \streetaddress{}
  \city{Bloomington} 
  \state{Indiana} 
  \postcode{47408}
}
\email{jmudvari@iu.edu}

% The default list of authors is too long for headers}
\renewcommand{\shortauthors}{B. Trovato et al.}


\begin{abstract}
This paper will explore how service providers in health-care indus-tries use data generated when patients provide information about their family history, medical history, food habit, exercise habit.
\end{abstract}

\keywords{}


\maketitle

\section{Introduction}

Health service providers collect high volume of information fromthe consumers every time they visit the facilities. These informa-tions or big data provides helpful insights for diagnostic purposeand treatment options. These data can range from clinical or patho-logical category to food and exercise habits, family history or per-sonal body mass index. Clinical practitioners require data to maketheir medical diagnosis, treatment recommendation, and prognosis.A richer set of near-real-time information can greatly help physi-cians determine the best course of action for their patients, discovernew treatment options, and potentially save lives [?]. So to speakfields big data applications in health care for the purpose of im-proving patient care is wide; disease prevention and management,health education, research and development, prognosis informa-tion sharing, public and individual health management, medical optimization.

Health data are stored as electronic medical records(EMR) whichare analyzed and shared among clinicians. These data are near realtime data. One of the trending example is application of big datain tackling opioid crisis in US. ”Data scientists at Blue Cross BlueShield have started working with big data experts at Fuzzy Logixto tackle the problem. Using years of insurance and pharmacy data,Fuzzy Logix analysts have been able to identify 742 risk factors thatpredict with a high degree of accuracy whether someone is at riskfor abusing opioids”[?].

\begin{acks}

  The authors would like to thank 

\end{acks}

\bibliographystyle{ACM-Reference-Format}
\bibliography{report} 

\end{document}
