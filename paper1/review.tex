\documentclass[sigconf]{acmart}

\usepackage{hyperref}

\usepackage{endfloat}
\renewcommand{\efloatseparator}{\mbox{}} % no new page between figures

\usepackage{booktabs} % For formal tables

\settopmatter{printacmref=false} % Removes citation information below abstract
\renewcommand\footnotetextcopyrightpermission[1]{} % removes footnote with conference information in first column
\pagestyle{plain} % removes running headers

\begin{document}
\title{Big Data Applications in Improving Patient Care}


\author{Janaki Mudvari Khatiwada}
\orcid{HID330}
\affiliation{%
  \institution{University of Indiana}
  \streetaddress{}
  \city{Bloomington} 
  \state{Indiana} 
  \postcode{47408}
}
\email{jmudvari@iu.edu}




\begin{abstract}
 Big data and its applications in providing best service outcome to the patients is a new trend. Patient care is the 
 main objective of healthcare organizations. Getting best possible care in terms of costs and service outcome are 
 patients' expectations. How service providers in health-care industries are using big volume of health related data 
 that are generated when patients provide information about their family history, medical history, food and exercise habit or 
 results from clinical tests? Besides health related data, patients are requested fill out a survey about their overall 
 experiences and even asked to give any recommendations to improve their service. 
\end{abstract}

\keywords{Big Data, Health Care, Patient care, Electronic health records}

\maketitle


\section{Introduction}
Health care is one of the service sectors where service providers claim to have provided consumers with the best experiences 
possible, whereas consumers are always researching for the best care facilities that they could possibly get which might save 
time and money and help them have a quality of life. Health service providers collect high volume of information from the 
consumers every time when they visit the facilities. The volume of health related informations generated in a high velocity 
is what consist of big data in health care sector.
These informations besides clinical records can be anything related to a person. Such as  person's ethnic background, exercise routine
and the time he/she spends on it on a weekly or daily basis, general daily meal the person intakes, records on wearable health devices 
and monitors. In today's world big data has become very impactful in policy making, solving problems and making prediction on 
whole range 
of areas. Healthcare industry has become one of the most important sectors to make of use of big data. Big data provides 
helpful insights 
for prevention, prediction, diagnosis and identification of best treatment option among all, on the basis of insurance plan a person 
has. Clinical practitioners acquire, share, compare and analyze big data trend to make their medical diagnosis, treatment
recommendation, and prognosis. A richer set of near-real-time information can greatly help
physicians determine the best course of action for their patients, discover new treatment
options, and potentially save lives \cite{www-hpe}.Consumers on the other hand use service provider's websites or web-pages 
to have an insight of the facilities and available physicians.
We look for the ratings and reviews in general based upon which we choose the facility and physician. Big data applications in 
health care for the purpose of
improving patient care is wide; disease prevention and management, health education, research and development, prognosis  
information sharing, public and individual health management, medical optimization. A goal of modern healthcare systems is to
provide optimal health care through the meaningful use of health information technology in order to improve health care quality 
and coordination, so that outcomes are consistent with current professional knowledge \cite{www-mapr-com}.



\section{Applications}

Health data are stored as electronic medical records(EMRs),electronic health records(EHRs) or any unstructured records, which are 
analyzed and shared among clinicians. These data are near real time data. The EHR, being adopted in many countries, offers a source 
of data the depth of which is almost inconceivable. About 500 petabytes of data was generated by the EHR in 2012, and by 2020, the 
data will reach 25,000 petabytes \cite{www-ghdonline-org}. One of the trending examples of application of big data in tackling 
opioid crisis in US.
Data scientists at Blue Cross Blue Shield have started working with big data experts at Fuzzy Logix to tackle the problem. Using 
years of insurance and pharmacy data, Fuzzy Logix analysts have been able to identify 742 risk factors that predict with a high 
degree of accuracy whether someone is at risk for abusing opioidsb \cite{www-datapine-com}. ZEO, Inc. is analyzing over a million 
nights of data to help consumers improve their sleep \cite{www-ghdonline-org}.

In general, applications of big data in health care for improving patient care can be categorized into following categories: 
Prevention, Prediction, Diagnosis, Disease Management and Research and Development.

\subsection{Prediction}

Analysis of available health records help make prediction which ultimately benefits general population. Making predictions is one of 
the most useful applications of big data.Researchers use analysis of medical records to make prediction of patients at risk to 
a disease. The United States National Institutes of Health has a project known as Pillbox, in which big data are used through 
the National Library of Medicine \cite{www-tandf-com}. Johns Hopkins University (Baltimore, MD, USA) developed a disease 
prediction system using the social media service Twitter \cite{www-ncbi-nlm-nih-gov}
The Seton Healthcare Family (Austin, TX, USA) and IBM Joint Development Program have analyzed and tracked medical information, and
have predicted outcomes of two million patients per
year\cite{www-uhcjsc-com}. Prediction models are especially useful in explaining epidemics and finding the best approach to deal 
with it. This helps in population health management. Optum Labs has collected EHRs of over 30 million patients to create a database 
for predictive analytics tools that will help doctors make big 
data-informed decisions to improve patients treatment \cite{www-mapr-com}. In another study, Parkland Health and Hospital System 
in Dallas, Texas, has developed a validated EHR-based algorithm to predict readmission risk in patients with heart failure.
Patients deemed at high risk for readmission receive evidence-based interventions, including education by a multidisciplinary
team, follow-up telephone support within two days of discharge to ensure medication adherence, an outpatient follow-up 
appointment within seven days, and a non-urgent primary-care appointment \cite{www-google-com}.


\subsection{Prevention}

The mantra, "Prevention is always better than cure" is what everybody
wants to follow.Till now physicians have been studying the general pattern of people's lifestyle and make a recommendations on 
keeping as it is or make a change to prevent their patients from any health problems.Big data help them identify vulnerable 
population and raise awareness. For example, physicians recommend general public to watch their weight in order to prevent them 
from diabetes and heart disease. Another such example is, physicians have identified certain population of certain race are more
prone to skin cancer when exposed to sun's ultraviolet rays while other race is more prone to have breast cancer. So, they 
raise awareness and make needed recommendations accordingly. This in totality help make general public's life better and help them live
longer and healthy life. Now we have smart-phones and wearables to track our fitness in general, which generate huge volume of data 
at a high velocity. In the near future, physicians might be using these data to have an understanding of any potential problem 
and prepare them for necessary remedies. Often by partnerships between medical and data professionals, with the potential to peer 
into the future and identify problems before they happen\cite{www-forbes-com}. One recently formed example of such a partnership is 
the Pittsburgh Health Data Alliance - which aims to take data from various sources (such as medical and insurance records, 
wearable sensors, genetic data and even social media use) to draw a comprehensive picture of the patient as an individual, in order 
to offer a tailored healthcare package \cite{www-forbes-com}. 100Plus uses public and private data to motivate consumers to take 
small healthy steps to change daily habits via a mobile application \cite{www-ghdonline-org}.  

 

\subsection{Diagnosis}
Early diagnosis of a disease helps in early intervention of disease management thereby saving lives and reducing costs. 
Prediction models developed by researchers by using big data help in early diagnosis.Predictive modeling over data derived 
from electronic health records(EHRs) is being used for early diagnosis and is reducing mortality rates from problems such as c
ongestive heart failure and sepsis \cite{www-mapr-com}.

\subsection{Disease Management} 
Wearable sensors, monitors and other smart devices help both caregivers and patients to keep track of any changes in factors 
that is affecting their health. Processing real-time events with machine learning algorithms can provide physicians with insights 
to help them make lifesaving decisions and allow for effective interventions \cite{www-mapr-com}.Ideally, individual and population 
data would inform each physician and his orher patient during the decision-making process and help determine the most 
appropriate treatment option for that particular patient \cite{www-link-springer-com}. 



\subsection{Research and Development}
Big data from past help physicians identity general variables responsible for illnesses. After identifying general trend, 
they can make precise recommendation to their patients and thereby help them have a quality of life and save them costs. 
Research and development is one of the important applications of big data and
analytics that helps in finding new tools, more effective medications, drugs 
and treatment regimen.Data-sharing arrangements between the pharmaceutical giants has led to breakthroughs such as the discovery 
that desipramine, commonly used as an anti-depressant, has potential uses in curing types of lung cancer \cite{www-forbes-com}. 
Big data helps Pharmaceuticals reduce cost of research and therefore lowers drugs cost which benefits patients. Data from 
clinical trials and patients records help identify adverse effects of a drug.
 
 




\section{Challenges}
While big healthcare data and applications and analytics provides a huge opportunity in improving patient care, it equally 
comes with some challenges. 
Privacy and security of personal information is one of the biggest challenges.
In February of 2015, the largest ever healthcare-related data theft took place, when hackers stole records relating to 80
million patients 
from Anthem, the second largest US health insurer \cite{www-forbes-com}.
Since healthcare data are large in volume and are in variety of forms; structured or unstructured, managing this big data of 
such variety is a challenge. Transforming big volume of unstructured data data which comes in such a velocity, into structured
version is another challenge. Data sharing between institutions is another challenge. Maintaining privacy of people's records 
can be a huge liability for the organizations involving in information sharing.




\section{Conclusion}
While big data in healthcare has some challenges, it still has some very important implications.Physicians and analysts are
using big data in prevention, prediction, diagnosis and disease management. With the help of big data physicians are able to 
identify disease risk factors and may predict a problem if those risk factors are not resolved. Thereby making needed 
recommendations which will ultimately help a patient. Discussions above present few examples of applications of big data for 
improving patient care. 
There is an increasing trend in making use of patients' clinical records for  analytics.  
Going through literatures indicates that use of big data in improving patient care is in the beginning phase. Information technology 
has provided consumers with variety of wearables making people more conscious about their health.In near future physicians might 
make use of data from the wearables to have an understanding of patients health. Health insurance companies might use big streaming 
data from wearables to provide incentive such as lowering  insurance premium or rewards point to people who are consistent in exercising.


\begin{acks}

I would like to thank prof.Gregory Von Laszweski and teaching assistants who helped me throughout my writing. 
\end{acks}

\bibliographystyle{ACM-Reference-Format}
\bibliography{report} 

\section{Bibtex Issues}
\section{Issues}

\DONE{Example of done item: Once you fix an item, change TODO to DONE}

\subsection{Uncaught Bibliography Errors}

    \TODO{Citations in text showing as [?]: this means either your report.bib is not up-to-date or there is a spelling error in the label of the item you want to cite, either in report.bib or in report.tex}

\subsection{Formatting}

    \TODO{Incorrect number of keywords or HID and i523 not included in the keywords}

\end{document}

\section{Bibtex Issues}
\section{Issues}

\DONE{Example of done item: Once you fix an item, change TODO to DONE}

\subsection{Uncaught Bibliography Errors}

    \TODO{Citations in text showing as [?]: this means either your report.bib is not up-to-date or there is a spelling error in the label of the item you want to cite, either in report.bib or in report.tex}

\subsection{Formatting}

    \TODO{Incorrect number of keywords or HID and i523 not included in the keywords}

\end{document}
