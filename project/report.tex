\documentclass[sigconf]{acmart}

\usepackage{graphicx}
\usepackage{hyperref}
\usepackage{todonotes}

\usepackage{endfloat}
\renewcommand{\efloatseparator}{\mbox{}} % no new page between figures

\usepackage{booktabs} % For formal tables

\settopmatter{printacmref=false} % Removes citation information below abstract
\renewcommand\footnotetextcopyrightpermission[1]{} % removes footnote with conference information in first column
\pagestyle{plain} % removes running headers

\newcommand{\TODO}[1]{\todo[inline]{#1}}

\begin{document}
\title{Big Data Analytics in Monitoring  Outdoor Air Quality}


\author{Janaki Mudvari Khatiwada}

\affiliation{%
  \institution{Indiana University, Bloomington}
  \streetaddress{P.O. Box 1212}
  \city{Bloomington} 
  \state{Indiana} 
 \postcode{43017-6221}
}
\email{jmudvari@iu.edu}





\begin{abstract}
Outdoor air pollution is one of the risk factors of public health. Air pollution adds burden to public health. Both developing and developed world use new technology and expertise to monitor outdoor air quality. 
United States Environmental Protection Agency (USEPA) collects outdoor air quality data from state, local 
and tribal agencies through outdoor air quality monitors across the country. The data get collected into the Air 
Quality System (AQS) database. This data can be used for variety of purposes such as education, research
and regulatory. Data from this data--mart is available for different time--series: hourly, daily, weekly, monthly 
and yearly. It gives us a real picture of outdoor air quality and measurements of pollutants present in the air in 
a particular time period. The data can be used for visualizations to see the trend of air pollutants at different time periods in a day or a season  and can also be combined with emissions data for comparative study. Since, air quality is vital for public health and 
environmental health, air quality monitoring possess great significance.


\end{abstract}

\keywords{i523, hid330, Outdoor Air Quality, Big Data, Particulate Matter }


\maketitle



\section{Introduction}
Outdoor air is a valuable natural resource that is vital to the health and existence of human beings and other forms of life. Several health research have revealed that air pollutants are contributing factors for lung cancer, 
cardiovascular disease, acute and chronic respiratory conditions. World Health Organization (WHO) in 2013 has assessed that air pollution is carcinogenic to humans \cite{www-who}. ``In 2012 WHO estimated that 72 percent of outdoor air pollution-related premature deaths were due to ischaemic heart disease and strokes'' \cite{www-who}.
Being aware of this fact, governments along with the scientists and the environmentalists help make policies to combat air pollution. Each country has set their own standards for outdoor air quality to protect their citizen's health. Every nation's standards depend upon their economic, cultural, social and political needs. The United States enacted its Clean Air Act in 1970 and was amended in 1990 as a way to set stage for combating air pollution challenges \cite{epa-gov}. Since then, the country has made a lot of progress in improving air quality while sustaining a constant economic growth. Today, United States, European nations, India, China and so many other countries monitor outdoor air quality and use the collected data for identifying the particles present in the air, their contribution to various health problems, their sources, health research and also to find out the solutions to minimize their production level.

Thousands of air quality monitors located  across the united States including US Virgin Islands and Puerto Rico live
stream outdoor air quality data to a national air database system called Air Quality System (AQS). AS a result big outdoor air data is generated everyday.Air Quality System database is a national database where state, local and tribal agencies submit all of the
data collected from thousands of air quality monitors across the United States. AQS database also called AQS Data Mart, has summary of yearly air quality data since the year of 1957. These data give an understanding of outdoor air quality and different particles present in the air and their sources. Source of air particles can be natural or human generated. Pollen, smoke from wildfires, mold, dust are some of the natural air pollutants. Similarly, emissions from
power--plants, industries and vehicles, different substances and solutions that human have generated for various purpose are human generated air pollutants. 
Based on the value of Air Quality Index (AQI), USEPA has classified Outdoor air quality as 'Good', 'Moderate',
'Unhealthy for sensitive Groups', 'Unhealthy',
'Very Unhealthy' and 'Hazardous' \cite{airnow-gov}. The AQI value range from 0 to 500. The agency has assigned colors ('Green', 'Yellow', 'Orange', 'Red', 'Purple' and 'Maroon' respectively) to each of the air quality categories \cite{airnow-gov}.

   
 



\section{Big Data and Outdoor Air Quality}
In US there are about 4,000 outdoor air quality monitors operated by state environmental agencies \cite{outdoor-air}. They constantly collect air data on harmful suspended particles present in the air and send them to a national database center which is
AQS database. The data contains valuable information about the concentrations of different air pollutants in different time series; hourly, daily, weekly and yearly. Besides air quality data, AQS database also contains emissions data and weather data which are vital to the outdoor air air quality. Emissions data is basically data from vehicular and industrial emissions. They help to  understand the source of different air pollutants and their role in qir quality as well as they can be used for furthering research in limiting their emissions. Yearly summery data of AQS Data Mart can also be used to see the progress made in reducing harmful air pollutants over the years.
India and China, two bigger economies in the world are battling worst air pollution. 


\section{Air Pollutants, Particulate Matter}




\section{Air Quality Index}











\section{Conclusion}
There is tremendous application prospects of using MQTT protocol in big data and edge computing. Edge computing, being a new  development in data analytics,   


\begin{acks}

  The author would like to thank Dr. Gregor von Laszewski for his
  support and suggestions to write this paper.

\end{acks}

\bibliographystyle{ACM-Reference-Format}
\bibliography{report} 



\end{document}
